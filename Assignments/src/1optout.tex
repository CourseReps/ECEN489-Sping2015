\documentclass[11pt]{article}

%%  Dimensions and URL
\usepackage[margin=1in]{geometry}
\usepackage{hyperref}

%%  Definitions
\renewcommand{\baselinestretch}{1.1}
\pagestyle{plain}


\begin{document}

\begin{center}
{\bfseries \LARGE Opt-Out Challenge 1}
\end{center}

\paragraph{Rational:}
Engineering students possess vastly different programming skills and experience.
This situation makes it difficult for instructors to create activities from which everyone can benefit.
To address this situation, this course features two types of assignments.
The basic exercises focus on the fundamentals of the Java programming language, with elementary questions and complementary programming challenges.
Each opt-out assignment typically consists of a programming challenge that showcases a mastery of pertinent programming concepts.
On every occasion, students must elect to complete the basic assignment or the opt-out challenge, but not both.
Accomplished programmers are strongly encouraged to choose the latter option.
The opt-out challenge is graded as satisfactory (full credit) or unsatisfactory (no credit); a working prototype must be presented in class.


\paragraph{Reading Assignment:}
How to Program Java, 10th edition
\begin{itemize}
\item Chapter~1 -- Introduction to Computers and Java
\item Chapter~2 -- Introduction to Java Applications
\item Chapter~3 -- Introduction to Classes, Objects, Methods and Strings
\end{itemize}


\section*{Programming Challenges}

Create a \texttt{GatheredData} class that contains the following fields, along with appropriate methods.
\begin{itemize}
\item \texttt{time (long)}
\item \texttt{longitude (double)}
\item \texttt{latitude (double)}
\end{itemize}
Write a program that takes two objects (instances) of this class and computes the average velocity between the two points, assuming motion along the shortest path.
If you are not familiar with the haversine formula, you may want to look it up before you implement your program.

\begin{enumerate}
\item Implement this task in Java.
\item Using IntelliJ IDEA, Git, and GitHub, commit your code as a project labeled \texttt{Challenge1} under \texttt{Students/<GitHubID>/}, where \texttt{<GitHubID>} should be replaced by your username on \href{https://GitHub.com}{GitHub}.
\end{enumerate}


\end{document}
