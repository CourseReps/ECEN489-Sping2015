\documentclass[11pt]{article}

%%  Dimensions and URL
\usepackage[margin=1in]{geometry}
\usepackage{hyperref}

%%  Definitions
\renewcommand{\baselinestretch}{1.1}
\pagestyle{plain}


\begin{document}

\begin{center}
{\bfseries \LARGE Opt-Out Challenge 5}
\end{center}

\paragraph{Reading Assignment:}
How to Program Java
\begin{itemize}
\item Chapter~17 -- Files, Streams and Object Serialization
\item Chapter~18 -- Recursion
\item Chapter~19 -- Searching, Sorting and Big O
\item Chapter~20 -- Generic Collections
\item Chapter~21 -- Generic Data Structures
\item Chapter~22 -- Custom Generic Data Structures
\end{itemize}


\section*{Programming Challenge}

This programming challenge contains two components.
First, you must create a Java server that can receive information from clients.
Second, you must program an Android application that records local observations and sends them to the server when connected to a Wi-Fi network.

As in previous challenges, upon accepting a connection, the server should create a table for the remote client using \href{http://www.sqlite.org/}{SQLite}.
The server should subsequently enter every entry reported by the Android device into the database.
Again, you may want to leverage the Java Database Connectivity (JDBC) API in your implementation.

The Android app should have two modes of operation.
A sense mode where the device periodically records measurements into a local SQLite database.
And, a transfer mode that takes every new entry present in the local database and transfers it to the server.
Be sure to include a flag with every observation sequence to indicate whether this data was previously transferred to the remote database. 
If you need to borrow an Android device, we may be able to accommodate the first few requests.

\subsection*{Code}

\begin{enumerate}
\item Implement the server in Java.
\item Using IntelliJ IDEA, Git, and GitHub, commit your code for the server as a project labeled \texttt{Challenge5server} under \texttt{Students/<GitHubID>/}.
\item Using \href{http://developer.android.com/sdk/index.html}{Android Studio}, \href{http://git-scm.com/}{Git}, and \href{https://GitHub.com}{GitHub}, commit your application for the client as a project labeled \texttt{Challenge5android} under \texttt{Students/<GitHubID>/}.
\item As part of your demonstration, you will have to use an \href{http://www.sqlite.org/}{SQLite} visualizer to showcase the content of the table before and after the connection.
\end{enumerate}



\end{document}
