\documentclass[11pt]{article}

%%  Dimensions and URL
\usepackage[margin=1in]{geometry}
\usepackage{hyperref}

%%  Definitions
\renewcommand{\baselinestretch}{1.1}
\pagestyle{plain}


\begin{document}

\begin{center}
{\bfseries \LARGE Opt-Out Challenge 4}
\end{center}

\paragraph{Reading Assignment:}
How to Program Java, 10th edition
\begin{itemize}
\item Chapter~10 -- Object-Oriented Programming: Polymorphism
\item Chapter~11 -- Exception Handling: A Deeper Look
\item Chapter~16 -- Strings, Characters and Regular Expressions
\end{itemize}


\section*{Programming Challenges}

Using Android Studio, build an application that connects to a server and periodically reports the status of phone parameters such as
\href{http://developer.android.com/reference/android/location/LocationManager.html}{location},
\href{http://developer.android.com/reference/android/hardware/SensorManager.html}{orientation},
and \href{http://developer.android.com/reference/android/net/wifi/WifiManager.html}{Wi-Fi} attributes.
Program a Java server that can support this type of connection and displays the latest message it has received.

\paragraph{Coordination:}
At this point, several members have been successful at building server-client infrastructures.
This time, students who attempt this exercise must coordinate with one another to make sure that all the phone clients can connect with servers.
This can be achieved by jointly specifying an interface between the clients and servers.
Options for the mode of interaction between the agents include \href{http://www.json.org/}{JSON} objects, \texttt{serializable} objects, repeated queries from the server, etc.
This will be discussed in class to formalize and define the common interface.


\subsection*{Code}

\begin{enumerate}
\item Implement the server in Java and the client as an Android application.
\item Using \href{https://www.jetbrains.com/idea/}{IntelliJ IDEA}, \href{http://git-scm.com/}{Git}, and \href{https://GitHub.com}{GitHub}, commit your code for the server as a project labeled \texttt{Challenge4server} under \texttt{Students/<GitHubID>/}.
\item Using \href{http://developer.android.com/sdk/index.html}{Android Studio}, \href{http://git-scm.com/}{Git}, and \href{https://GitHub.com}{GitHub}, commit your application for the client as a project labeled \texttt{Challenge4android} under \texttt{Students/<GitHubID>/}.
\end{enumerate}


\end{document}
