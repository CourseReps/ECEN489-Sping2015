\documentclass[11pt]{article}

%%  Dimensions and URL
\usepackage[margin=1in]{geometry}
\usepackage{hyperref}

%%  Definitions
\renewcommand{\baselinestretch}{1.1}
\pagestyle{plain}


\begin{document}

\begin{center}
{\bfseries \LARGE Opt-Out Assignment 2}
\end{center}

\paragraph{Reading Assignment:}
How to Program Java, 10th edition
\begin{itemize}
\item Chapter~4 -- Control Statements: Part~1
\item Chapter~5 -- Control Statements: Part~2
\item Chapter~6 -- Methods: A Deeper Look
\item Chapter~27 -- Networking (Sections 27.4, 27.5, 27.6)
\end{itemize}


\section*{Programming Challenges}

Create a two-way chat service composed of a server and a client.
This should be implemented using stream sockets.
The client should be able to access the server over the Internet.


\subsection*{Simple Server Using Stream Sockets}

Establishing a simple server in Java requires five steps.

\begin{enumerate}
\item Create a \texttt{ServerSocket} object:
\begin{verbatim}
  ServerSocket server = new ServerSocket(portNumber, queueLength);
\end{verbatim}
The variable \texttt{portNumber} is an admissible TCP port number and \texttt{queueLength} is the maximum number of clients that can wait to connect to the server.
\item Wait for a connection:
\begin{verbatim}
  Socket connection = server.accept();
\end{verbatim}
In this step, the server listen indefinitely for an attempt by a client to connect.
The method returns a \texttt{Socket} when a connection with a client is established.
\item Manage the I/O streams associated with the socket:
\begin{verbatim}
  connection.getOutputStream();
  connection.getInputStream();
\end{verbatim}
These objects can subsequently be employed to send or receive bytes with the \texttt{OutputStream} method \texttt{write} and the \texttt{InputStream} method \texttt{read}, respectively.
One can also use classes such \texttt{ObjectInputStream} and \texttt{ObjectOutputStream} to enable entire objects to be read from or written to a stream, a technique called wrapping.
\item Support the live interaction:
In the processing phase, the server and the client communicate via the \texttt{OutputStream} and \texttt{InputStream} objects.
\item Closing the connection:
The server closes the connection by invoking the \texttt{close} method on the streams and on the \texttt{Socket}.
\end{enumerate}


\subsection*{Simple Client Using Stream Sockets}

Establishing a simple client in Java necessitates four steps.

\begin{enumerate}
\item Create a \texttt{Socket} to connect to the server:
\begin{verbatim}
  Socket connection = new Socket( serverAddress, port);
\end{verbatim}
When the connection attempt is successful, this returns a \texttt{Socket}.
\item Manage the I/O streams.
\item Support the live interaction.
\item Close the connection:
The client closes the connection by invoking the \texttt{close} method on the streams and on the \texttt{Socket}.
\end{enumerate}


\subsection*{Code}

\begin{enumerate}
\item Implement this task in Java.
\item Using IntelliJ IDEA, Git, and GitHub, commit your code for the server as a project labeled \texttt{Challenge2server} under \texttt{Students/<GitHubID>/}, where \texttt{<GitHubID>} should be replaced by your username on \href{https://GitHub.com}{GitHub}.
\item In a similar fashion, commit your code for the client as a project labeled \texttt{Challenge2client} under \texttt{Students/<GitHubID>/}.
\end{enumerate}


\end{document}
