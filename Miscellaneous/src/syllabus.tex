\documentclass[letterpaper,11pt]{article}

%%  Dimensions and URL
\usepackage[margin=1in]{geometry}
\usepackage{url}

%%  Definitions
\renewcommand{\baselinestretch}{1.1}
\pagestyle{plain}


\begin{document}

\begin{center}
{\LARGE \sc Special Topics in Cloud-Enabled Mobile Sensing\\[5mm]}
\end{center}

\begin{center}
\begin{tabular}{llll}
\textbf{ECEN:} & 489-501 & \textbf{CRN:} & 26434 \tabularnewline
\textbf{Lecture:} & MW 5:45 pm -- 6:35 pm & {Location:} & EIC \tabularnewline
\textbf{Laboratory:} & F 2:00 pm -- 5:00 pm & {Location:} & EIC \tabularnewline
\textbf{Instructors:} & Dr.~Jean-Francois Chamberland & \multicolumn{2}{l}{chmbrlnd@tamu.edu} \tabularnewline
& Dr.~Gregory Huff & \multicolumn{2}{l}{ghuff@tamu.edu} \tabularnewline
\textbf{Office Hours:} & F 2:00 pm -- 5:00 pm & {Location:} & EIC \tabularnewline
\end{tabular}
\end{center}

\paragraph{Course Description:}
The goal of this multidisciplinary project-based laboratory course is to provide instruction on system-level integration and design through a range of hands-on activities that complement the traditional classroom experience.
This includes the Android prototyping platform, Eclipse, Java, Git, SQL and cloud services.
The focus is on modular application development, algorithms, information management, storage and data visualization.
In addition, emphasis is put on team work, presentation skills, time management, creativity and innovation.

\paragraph{Objectives:}
\begin{enumerate}
\topsep0.3ex plus 0.2ex
\parsep0.2ex plus 0.2ex
\labelwidth0em
\labelsep1em
\itemindent0em
\leftmargin\labelwidth
\rightmargin0pt
\itemsep0ex
\listparindent0em
\item
Enhance engineering education by facilitating learning through engineering projects.

\item
Review basics of project development, programming concepts, the fundamentals of system design.

\item
Foster leadership and team work, with division of labor, complementary tasks, discussion and integration.

\item
Develop the ability to bridge theoretical concepts and practical tasks.

\item
Master elements of experiential learning: abstract conceptualization, active experimentation, concrete experience, reflective observation.

\item
Improve transferable engineering skills and the ability to integrate different concepts.

\item
Develop confidence and leadership.

\item
Promote creativity and critical thinking.

\item
Refine presentation skills and the ability to conduct and manage projects.
\end{enumerate}

\paragraph{Assignment:}
Assignments, bi-weekly mini-projects and tutorials should be anticipated by participants.
Students are encouraged to work in groups for many assignments, but required to complete certain tasks independently.
Coded solutions must be submitted using Git and GitHub, a distributed revision control and source code management system.

\paragraph{Recommended Texts:}
There exist many books that offer an excellent treatment of the technologies surveyed in this course.
Several such books are available at the library:
\begin{quote}
\url{http://library.tamu.edu}.
\end{quote}
\begin{center}
\begin{tabular}{ll}
\emph{Java How to Program, 10th edition} & Paul Deitel, Harvey Deitel \tabularnewline
\emph{Java, A Beginner's Guide, 5th edition} & Herbert Schildt \tabularnewline
\emph{C++ Primer, 5th edition} & Stanley B. Lippman, Jos\'{e}e Lajoie, Barbara E. Moo
\end{tabular}
\end{center}

\paragraph{Grade Components:}
The major grade components for ECEN 489, and their weights, are listed below.
Assignment and test grades will only be discussed after class or during office hours.
We reserve the right to ask students to present their concerns or arguments in writing.
Failure to meet a deadline may result in a grade of zero for the corresponding work.
\begin{center}
\begin{tabular}{lc}
Assignments, Labs, and Tasks & 30 \% \\
Participation, Presentations, and Media & 20 \% \\
Projects & 50\%
\end{tabular}
\end{center}
If your overall grade falls within one of the prescribed ranges, then you are guaranteed to receive at least the letter grade indicated.
\begin{center}
\begin{tabular}{lclc}
A & 90 -- 100 \% & D & 60 -- 69 \% \\
B & 80 -- 89 \% & F & 0 -- 59 \% \\
C & 70 -- 79 \%
\end{tabular}
\end{center}
The Academic Rules website at Texas A\&M University, and its section on Grading in particular, discusses possible grades and their respective meaning:
\begin{quote}
\url{http://student-rules.tamu.edu/rule10}.
\end{quote}

\paragraph{Attendance Policy:}
The University views class attendance as the responsibility of an individual student.
Attendance is essential to complete the course successfully.
University rules related to excused and unexcused absences are located on-line at:
\begin{quote}
\url{http://student-rules.tamu.edu/rule07}.
\end{quote}
Recitation sessions will be devoted to reviewing concepts from class, providing additional examples and real-world applications.
Partial answers to assigned problems will be discussed, together with interactive question-and-answer sessions during which students will be given a chance to collectively solve problems at the board and exchange freely with their instructor.

\paragraph{Academic Integrity:}
\emph{``An Aggie does not lie, cheat, or steal, or tolerate those who do.''}
Upon accepting admission to Texas A\&M University, individuals immediately assume a commitment to uphold the Honor Code.
You are expected to know and to follow the Aggie Honor Code.
Please think about what the Honor Code means, and let it shape and guide your behavior.
You are referred to the Aggie Honor System Office for more information:
\begin{quote}
\url{http://aggiehonor.tamu.edu}.
\end{quote}
It is the mission of the Aggie Honor System Office to serve as a centralized system established to respond fairly to academic violations of the honor code at Texas A\&M University.

\paragraph{Americans with Disabilities Act:}
The Americans with Disabilities Act (ADA) is a federal anti-discrimination statute that provides comprehensive civil rights protection for persons with disabilities.
Among other things, this legislation requires that all students with disabilities be guaranteed a learning environment that provides for reasonable accommodation of their disabilities.
If you believe you have a disability requiring an accommodation, please contact Disability Services, in Cain Hall, Room B118, or call 845-1637.
For additional information visit:
\begin{quote}
\url{http://disability.tamu.edu}.
\end{quote}

\paragraph{Classroom Communication Concerns:}
A student desiring to report a classroom communication concern should initiate the process within the first twelve class days of the semester, whenever possible, in order to identify an alternative course, if necessary.
The last date a student may initiate the classroom communication concerns procedure is the same as the Q-drop deadline.
For more information and forms, see:
\begin{quote}
\url{http://registrar.tamu.edu/forms/UGClsrmCommConcern.pdf}.
\end{quote}

\paragraph{Miscellaneous:}
Student dress, behavior, and speech are expected to be courteous and professional.
Any deviation from this deemed inappropriate by the professor or any disruptive behavior will result in immediate ejection from the class period with swift and appropriate disciplinary measures.


\subsection*{Course Outline \& Hours}

The course schedule and syllabus are tentative and subject to change.

\begin{center}
\begin{tabular}{|c|p{8cm}|c|}
\hline
Unit & Topics & Hours \tabularnewline
\hline
1 & Project Design and Implementation & 4 \tabularnewline
\hline
2 & Java and Eclipse & 8 \tabularnewline
\hline
3 & Network Programming & 4 \tabularnewline
\hline
4 & Android Programming & 8 \tabularnewline
\hline
5 & Communication Protocols & 4 \tabularnewline
\hline
6 & Data Management & 4 \tabularnewline
\hline
7 & Circuits and Sensors & 4 \tabularnewline
\hline
8 & Design Studio & 4 \tabularnewline
\hline
9 & Presentations and Media & 2 \tabularnewline
\hline
& \textbf{Total} & 42 \tabularnewline
\hline
\end{tabular}
\end{center}


\subsection*{Philosophy of Learning}

During the semester, we try to be available to students and to help them understand all of the material covered in class.
We also provide early feedback to people who may be in trouble, or may not get the final grade they desire.
This gives them an opportunity to learn more and to prepare better for exams.
We do realize that our classes are demanding and that students come with different backgrounds.
We try to minimize the impact of previous experience by focusing on basic material at the beginning of each semester.
We also offer office hours and optional review sessions to students.
These strategies are intended to give all students an equal chance at doing well.
Still, final grades are determined numerically based solely on individual standing.
This seems to be the only fair procedure to assign grades.
Alternate letter assignments with special consideration lead to favoritism.
Thus, final grades only reflect how well students did on their assignments, projects, and exams.
Unfortunately, they do not always reflect the amount of work and time invested in the class.
This is the nature of learning.
Ultimately grades are assigned fairly, if not pleasantly.
They are therefore very unlikely to change, unless we made a mistake in grading exams or adding numbers.

\end{document}

