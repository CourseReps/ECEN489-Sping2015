\documentclass[11pt]{article}

%%  Dimensions and URL
\usepackage[margin=1in]{geometry}
\usepackage{hyperref}

%%  Definitions
\renewcommand{\baselinestretch}{1.1}
\pagestyle{plain}


\begin{document}

\begin{center}
{\bfseries \LARGE Task 4 -- Android Development and Studio}
\end{center}


\section{Android}

Android is an operating system designed primarily for touchscreen mobile devices.
This project is driven by Google, and the source code is released under an Apache License.
This platform has a large community of developers creating applications that extend the functionality of devices, and these applications are written primarily in Java.
Android Studio is the official IDE for Android application development, and it is based on IntelliJ IDEA.
Among its many components, the Android Studio includes an IDE, the Android SDK tools, an Android platform and an emulator system image.
Basically, it leverages the strengths of IntelliJ, and adds the necessary software to rapidly build new applications.

\subsection*{Action Items}

\begin{itemize}
\item \textbf{Peruse:} Android Design and Develop.\\
\url{http://developer.android.com/design/} \\
\url{http://developer.android.com/develop/}
\item \textbf{Download and Install:} Android Studio.\\
\url{http://developer.android.com/sdk/}
\item \textbf{Download:} Samples.\\
\url{http://developer.android.com/samples/}
\end{itemize}


\section{Getting Started}

Having completed the preparation of your programming platform, you are now ready to create your first applications.

\subsection*{Action Items}

\begin{itemize}
\item \textbf{Read and Complete:} Building Your First App.\\
\url{http://developer.android.com/training/} \\
Complete the four stages of the Building Your First App tutorial -- Creating an Android project, Running your application, Building a simple user interface, Starting another activity.
\item \textbf{Commit:} Using Android Studio, Git, and GitHub, commit your code as a project labeled \texttt{Android1} under \texttt{Students/<GitHubID>/}, where \texttt{<GitHubID>} should be replaced by your username on GitHub.
\item \textbf{Read:} Saving Data.\\
\url{http://developer.android.com/training/}
\end{itemize}


\end{document}

